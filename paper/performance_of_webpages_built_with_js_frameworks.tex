\documentclass[a4paper, fontsize=11pt]{article}
\usepackage[english, ngerman]{babel}
\usepackage[autostyle]{csquotes}
\usepackage{natbib}

% \bibliographystyle{ksfh_nat}
% \bibliographystyle{apa}
\bibliographystyle{apalike}
% \bibliographystyle{plainnat}

\usepackage{hyperref}
\usepackage{graphicx}
\usepackage{titlesec}
% \usepackage{subcaption}
% \usepackage[raggedrightboxes]{ragged2e}
% \usepackage{pgf-pie}
% \usepackage{pgfplots}
% \usepackage[acronym, toc, numberedsection]{glossaries}
% \usepackage{pgfplots}

\usepackage{hyphenat}
% \hyphenation{Mathe-matik wieder-gewinnen}
\usepackage[T1]{fontenc}

% \makenoidxglossaries
% \newacronym{bs}{BS}{Radio Base Station}

\title{WIP: Mega-fast or just super-fast? Performance differences of mainstream JavaScript frameworks for web application}
\author{Andreas Nicklaus}

\makeatletter
\let\Title\@title
\let\Author\@author

\makeatother

\begin{document}

\begin{titlepage}
	
  \begin{center}

    \includegraphics[width=150px, keepaspectratio]{img/hdm-logo.png}
    
    \vspace{30px}
    {\large Masterarbeit im Studiengang Computer Science and Media}
    
    \vspace{20px}
    {\LARGE \Title}
    
    \vspace{20px}
    \noindent\rule{\textwidth}{1pt}
    
    \vspace{15px}
    vorgelegt von
    
    \vspace{10px}
    \textbf{{\large \Author}}
    
    % \vspace{5px}
    Matrikelnummer 44835
    
    \vspace{10px}
    an der Hochschule der Medien Stuttgart
    
    \vspace{10px}
    am \today
    
    \vspace{10px}
    zur Erlangung des akademischen Grades eines Master of Science
  \end{center}	
  
  \vfill
  
  \begin{tabular}[t]{ll}
    Erst-Prüfer: & Prof. Dr. Fridtjof Toenniessen \\
    Zweit-Prüfer: & Stephan Soller
  \end{tabular}		
\end{titlepage}

\selectlanguage{ngerman}

\section*{Ehrenwörtliche Erklärung}
	
	Hiermit versichere ich, \Author, ehrenwörtlich, dass ich die
	vorliegende Masterarbeit mit dem Titel: „\Title“ selbstständig und ohne fremde Hilfe verfasst und keine
	anderen als die angegebenen Hilfsmittel benutzt habe. Die Stellen der Arbeit, die dem
	Wortlaut oder dem Sinn nach anderen Werken entnommen wurden, sind in jedem Fall
	unter Angabe der Quelle kenntlich gemacht. Die Arbeit ist noch nicht veröffentlicht oder
	in anderer Form als Prüfungsleistung vorgelegt worden.\\
	
	Ich habe die Bedeutung der ehrenwörtlichen Versicherung und die prüfungsrechtlichen
	Folgen (§26 Abs. 2 Bachelor-SPO (6 Semester), § 24 Abs. 2 Bachelor-SPO (7 Semester), §
	23 Abs. 2 Master-SPO (3 Semester) bzw. § 19 Abs. 2 Master-SPO (4 Semester und
	berufsbegleitend) der HdM) einer unrichtigen oder unvollständigen ehrenwörtlichen
	Versicherung zur Kenntnis genommen.
	\vspace{30px}
	
	Eislingen, den \today
	\vspace{20px}
	
	\includegraphics[height=60px]{img/unterschrift.png}
%	\vspace{60px}
	\vspace{10px}
	
	\Author

\pagebreak

\selectlanguage{ngerman}
\begin{abstract}
  Diese Arbeit kurz und knackig.
  % TODO
\end{abstract}

\selectlanguage{english}
\begin{abstract}
  This work in a nutshell.
  % TODO
\end{abstract}

\vfill

\noindent\textbf{Dislaimer:} This paper has been written with the help of AI tools for translating sources and outlining parts of the written content.
All content has been written or created by the author unless marked otherwise.

\pagebreak

\tableofcontents
\pagebreak

\section{Introduction}\label{sec:introduction}
Throughout the evolution of the world wide web, many changes have disrupted the way websites are created.
From simple file servers run by few selected institutions, simple static web pages and dynamic services like blogs and forums to websites created with the help UI tools and web development frameworks, mainly written in JavaScript, development has changed drastically since its beginning.

Older web pages often lacked features, that developers today work with as a matter of course.
Yet their load and rendering most likely would be brazingly fast with today's technological advancements in networking, browser functionalities and user equipment.
Modern websites though are often bigger in size, have a lot more features and are in many respects more complex.
Due to the increased complexity, the mere volume of a webiste's data has increased, especially with more and more multimedia files
That in return has increased the demand for better performance on all components of the load and rendering process.
This technological advancement has upped the technological sophistication for development tools as well.
Today's modern web development frameworks support developers with tools to create sites and applications through terminal commands.
They often increase the content-per-line-of-code quota through implicit page generation in contrast to the explicit writing of source code from earlier times.
Many frameworks even feature configuration options for directly hosting the webpage.

As the generation process changed from writing code manually to automatically, this implicit page generation undoubtedly increased speed through faster content generation and a greater developer experience for some developers.
Because developer experience varies between different frameworks and some approaches are more intuitive to respective developers, a current trend has evolved for developers to become experts in a single framework rather than many.
This trend leads to a tribal conflict as to which framework is better than others with each tribe being convinced that their framework is the best.
There is no apparant way to determine a \enquote{best framework} in terms of Developer Experience because it is a subjective criterion.
The performance of a framework as assessed by the developer can be similar or greatly different, depending on the frameworks and the interviewees.

When it comes to User Experience and especially the Perceived User Experience however, there are plentiful collections of metrics and criteria to choose from so as to determine the performance of websites, not frameworks.
The optimization of websites has become a goal during development because it has a real effect on both the ranking of web pages in search engines and the user behavior.
Both effects create business interests and financial incentives to invest resources into performance optimization.
However, the lack of research on the topic suggests either a consensus for a negligible effect of the development framework on the website's performance or a lack of knowledge of the effect.
Measurements on the effect of the development framework are a major convoluted task simply because the performance of a specific website can be dependent on many other factors such as the user's device, browser, networking hardware or server-side hardware.
The number of possible combinations of factors and their reliability makes it difficult to measure a single performance run with a reliable result.
Every single result is only a small part of a large number of possible performances the same application could achieve with different parameters.
It is therefore perceivable that a \enquote{perfect combination} of hard- and software exists for each framework or in general, but it is currently not possible to find such a combination because the necessary data is missing.

Many modern web tracking services provide data about the user, the user's devices, current page load times and so on.
This data is helpful in determining current poor performances and therefore possible starting points for optimization efforts.
But it gives very little information about recommended actions or recommended choice of framework for a redesign of a web application.
Relying on marketing material for choice of framework is equally questionable because most modern frameworks claim to be fast, easy to use and performance efficient.
This suggests that each would be a great choice for developers.

In order to find a suitable framework for an application, a set of metrics needs to be at least outlined for comparison.
Many former studies suggest metrics to be relevant for the User Experience or Search Engine Optimization.
Content metrics such as word count or presence of meta tags might be important for some performance measurements, but might also have no effect on the User Experience.
In contrast, rendering metrics such as page load time or page weight might be ascribed to the framework used during development.

The performance of a framework towards the user can very rarely be compared because there are no publicly available comparisons between exact replicas of web applications built with different frameworks.
Therefore, a comparative study between the same website built with different frameworks is needed to get as close as possible to an exact website replica.
With this data, an informed choice might be made for other projects.

The goals of this paper are to propose a set of metrics that allow comparing mainstream JavaScript frameworks for web applications, to provide a comparative study between selected frameworks and create a tool to compare the rendering performance of a page as a whole and of dynamic components within a page.

\section{Related Work}\label{sec:relatedwork}
\section{Design}\label{sec:design}
\subsection{Example Application}\label{subsec:exampleapplication}
\subsection{Hosting Environments}\label{subsec:hostingenvironments}
\subsection{Testing Tools}\label{subsec:testingtools}

\section{Implementation}\label{sec:implementation}
\subsection{Components}\label{subsec:components}
\subsection{Tests}\label{subsec:tests}

\section{Evaluation}\label{sec:evaluation}
\subsection{Page Load Times}\label{subsec:pageloadtimes}
\subsection{Component Load Times}\label{subsec:componentloadtimes}
\subsection{Component Update Times}\label{subsec:componentupdatetimes}

\section{Conclusion}\label{sec:conclustion}
\section{Summary}\label{sec:summary}

\pagebreak

% \appendix
% \glsaddall
% \printnoidxglossary[type=\acronymtype,nonumberlist,style=long]

% \cite{5GEfficiencyOverview}

\nocite{*}
\bibliography{sources}{}
\end{document}